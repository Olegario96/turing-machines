\documentclass{article}
\usepackage[utf8]{inputenc}

\title{Trabalho 1 - Teoria da Computação}
\author{Gabriel Leal C. Amaral, Gustavo F. Olegário}
\date{October 2016}

\begin{document}
\maketitle
\section{Máquina 1a - Fita única}
    \subsection{Enunciado da linguagem}
        \text{L = \{w\#w / w \in{\mbox{\{a,b,c\}* }}\}}
    \subsection{Descrição do algoritmo}
        \begin{enumerate}
        \item No estado inicial, se ler a entrada vazia, aceite a palavra. Caso contrário, vá para o próximo estado.
        \item Marcar a posição lida com x caso seja a ou b.
        \item Percorrer até o primeiro elemento após a cerquilha que seja diferente de x.
        \item Caso esse elemento seja diferente do último marcado com x no lado esquerdo da cerquilha, rejeite. Caso contrário, marque com x e volte até o último elemento marcado no lado esquerdo da cerquilha. Se o elemento seguinte for x, aceite a palavra. Caso seja a ou b, vá para 2. Caso contrário, rejeite.
        \end{enumerate}
\section{Máquina 2b - Multifita}
    \subsection{Enunciado da linguagem}
        \text{L = \{ w$w^{R}$w / w \in{\mbox{\{a,b\}* }}\}}
    \subsection{Descrição do algoritmo}
        \begin{enumerate}
        \item No estado inicial, se ler a entrada vazia, aceite a palavra. Caso contrário, vá para o próximo estado.
        \item Copie cada entrada lida da fita 1 para a fita 2. Ao chegar na entrada vazia, volte uma posição à esquerda em ambos os cabeçotes.
        \item Mova o cabeçote da fita 1 para esquerda. Mova o cabeçote da fita 2, 3 vezes para a esquerda.
        \item Repita o passo anterior até o cabeçote 2 chegar à entrada vazia. Chegando na entrada vazia, mova o cabeçote das fitas 1 e 2 para a direita. Neste ponto, o cabeçote 1 aponta para o início da última string e o cabeçote dois para o começo da primeira.
        \item Percorrer até o final da string analisada na fita 1. Caso as entradas sejam iguais, marcar com x ou y (a ou b). Caso contrário, rejeite. Continue até a entrada vazia.
        \item Mova o cabeçote uma posição à esquerda na fita 1 (neste momento, o cabeçote da fita 1 irá apontar para o final da última string) e mantenha o cabeçote 2 onde está (neste ponto, ele está no começo da string reversa).
        \item Percorra a fita 1 no sentido direita-esquerda e a fita 2 no sentido esquerda-direita. Caso as entradas sejam iguais, continue até atingir uma entrada a ou b na fita 1 (alertando que acabou a string). Atingindo esta marca, aceite a palavra.
        \end{enumerate}
\end{document}

